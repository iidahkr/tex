\chapter{サービス運営}

\section{運営概要}
2014年8月に下旬に運営を開始してから、約3ヶ月のサービスの運営についての振り返りを実施した。当該期間に本サービスを用いて3試合がマッチングされ、79名に使用してもらった。
当期間の試合は下記3点の項目を検証するため、チームの抽出条件を下記のように設定した。
\begin{itemize}
	\item チーム数:9チーム (お互い試合をしたことがないチーム同士)
	\item 参加者の性別:男性のみ
	\item 職業:大学生:社会人
	\item 年齢層:20-25歳
\end{itemize}



当期間で下記3点の仮説を検証した。
\begin{itemize}
	\item[$H1$] 自分たちだけで試合を運営できるのか
	\item[$H2$] 本サービススタッフがコートに行かなくても試合を運営できるのか
	\item[$H3$] どれくらいのチーム数と試合時間が適切か
\end{itemize}






\section{2014年10月までの運営振り返り}

2014年8月に下旬に運営を開始してから、9月3日までの約1ヶ月のサービスの運営についての振り返りを実施した。当該期間に本サービスを用いて3試合がマッチングされ、33名に使用してもらった。
\\
\\
\\\textbf{【ファーストテスト( 8月30日)】}
\\1度目のテストを行い、検証項目である「自分たちでだけで試合が運営できるのか」という項目を検証した。
3チームが参加し、7分の試合を2時間行った。各参加者の詳細は表のとおりである。
\\ 事前に参加者には参加者のでタイムキーパーと審判を決めて回すように伝えておいた。

\begin{table}[htb]
	\begin{center}
		\caption{ファーストテスト (2014年8月30日)}
		\begin{tabular}{|l|c|r||r|} \hline
			チーム人数 & 活動頻度 & 値段についての感想 & 試合時間についての感想 \\ \hline \hline
			5人(社会人のみ) & 月一度程度 & ちょうどよい & 長く感じた \\
			7人(学生のみ) & 週2回 & 安い & ちょうどよい \\
			7人(学生3人) & 週2回 & ちょうどよい & ちょうどよい \\ \hline
		\end{tabular}
		\label{tab:price}
	\end{center}
\end{table}

タイムキーパーや審判がいなくても運営は成功したが、当日コートに到着してもどこのコートでやるのかわからなくて試合の時間が遅れてしまった。理由は2点ある。1点目はどこのコートで行われるかわからなかったためである。よって事前に送るメールに一度受付に行って当日のコートを確認してもらう必要がある。また当日の準備するチームと片づけをするチームが明確でなかったため、コートの準備に遅れが生じた。コートのビブスやボールも片づけられることがなかった。
\\
\\
\\\textbf{【セカンドテスト( 9月21日)】}
\\
\\
2度目のテストを行い、検証項目である「本サービスのスタッフが当日行かなくても運営できるか」という項目を検証した。
2チームが参加し、7分の試合を2時間行った。各参加者の詳細は表のとおりである。
\\ 受付で当日の試合が行われるコートと準備と片づけについて指示した。

\begin{table}[htb]
	\begin{center}
		\caption{セカンドテスト (2014年9月21日)}
		\begin{tabular}{|l|c|r||r|} \hline
			チーム人数 & 活動頻度 & 値段についての感想 & 試合時間についての感想 \\ \hline \hline
			13人(学生のみ) & 週3回 & ちょうどよい & ちょうどよい \\
			15人(学生のみ) & 週2回 & ちょうどよい & ちょうどよい \\ \hline
		\end{tabular}
		\label{tab:price}
	\end{center}
\end{table}
事前に送ったメールのマニュアル通り運営したため
後日、参加者にインタビューをしたところいつものように試合ができたという意見を貰った。
\\コートの片づけもされていたので、コート側の余計な労働はなかった。
\\
\\\textbf{【ファーストテスト( 9月27日)】}
4チームで7分の試合を2時間行った。チーム数を増やすことによって試合への満足度に変化が見られるのか調べた。

\begin{table}[htb]
	\begin{center}
		\caption{ファーストテスト (2014年9月27日)}
		\begin{tabular}{|l|c|r||r|} \hline
			チーム人数 & 活動頻度 & 値段についての感想 & 試合時間についての感想 \\ \hline \hline
			5人(社会人のみ) & 月一度程度 & ちょうどよい & 長く感じた \\
			5人(社会人のみ) & 月二回程度 & ちょうどよい & 長く感じた \\
			14人(学生のみ) & 週2回 & 安い & ちょうどよい \\
			8人(学生のみ) & 週2回 & 安い & ちょうどよい \\ \hline
		\end{tabular}
		\label{tab:price}
	\end{center}
\end{table}

待ち時間が長くなるので、7分の試合時間を5分程度にして、より多くの試合を回すことが重要であった。
2014年8月30日から1ヶ月の運営期間をもとに考察したLivenUpの成果と課題は下記のとおりである。

\subsection{【成果】}
・コートのアイドリングの時間を格安でフットサルチームが共有することでコートのアイドリングタイムで新たに稼ぐ手段ができ、需要があることを確認できた
\\
・事前のメールを送り、準備や片づけを指示したことで運営はスムーズにいった
\\
・最適なゲーム時間は3チームで7分がちょうどよかった
\subsection{【課題】}
・本サービス内のクーポンをすべて使用してしまった後に引き続き継続してもらうにはどうすればいいか。最初にクーポンを3枚購入したチームは3チームいたが、継続で買ってもらう際のクーポンの枚数はどのチームも1枚だった。
\\・本サービスにTwitterやFacebookを通して、新たなに参加したチームはいなかった。新規にユーザーを獲得する方法を模索する必要がある。
\\・アマチュアだけが参加することで、レベル感に不安を感じるユーザーはいなかったが、予めどのような属性 (学生、週の練習回数)のチームが来るなど事前に知らせることで不安を軽減する必要がある。
\\・本サービスを決済する際に、事前にチームメイトから集金でき、そのままシームレスに本サービスの決済を済ませることができればより決済周りの煩雑性は解消される。





\section{機能改善}
上記にもあるようにチーム内の集金のあり方を変えた。具体的には株式会社手嶋屋のpne.clubを用いり、クラブ内のメンバーからの集金を可能にすることで、代表者が一括で試合代金を負担することがないシステムを採用した。

