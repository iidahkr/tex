\chapter{結論と今後の課題}
\label{intro}
\section{運営総括}
2013年10月に本プロジェクトを構想して以降、株式会社手嶋屋の協力のもと、2014年8月にフットサルマッチングサイトであるLivenupをローンチすることができた。特に10月は月に3試合が組まれ、紹介を通じ、若干ではあるが新規会員が増えた。2013年12月までに8試合を催すことできた。今後は、徐々に本サービスの認知度を上げ、新規利用者の自然流入を図っていく必要がある。実際に2015年2月からは提携コートを増やし、店頭に本サービスのチラシを配布する予定である。
\\ サービスの運営を通じて「フットサルチームのマッチング」および「フットサルチームのアイドリングタイムの有効活用」を目的に、フットサルチームに格安でコートを提供しフットサルコートのアイドリングタイムを削減に成功した。これまでフラッシュマーケティングサイトがレジャーカテゴリーにおいてできていなかったシステムを構築し、新たに利用者へのニーズ喚起させ、満たしたと言える。ただ一方、また運営上の課題も多々あげられるので、次節以降で述べる。
\section{運営成果}
本サービスはインタビュー調査をもとに、従来のフラッシュマーケティングの問題点を洗い出し、サービスの仕組みを構築した。複数チームで共有することでコートのアイドリングタイムを活用できることがわかった。
\\ 本サービスが集金し、先払いを利用者に強いることで、従来フットサルコートが問題視していた当日のキャンセルのリスクは軽減され、アイドリングタイムで収入を得ることができた。
\\ この仕組みはフラッシュマーケティングが対象としているスポーツ施設に応用できる可能性がある。具体的にはテニスコートやゴルフ場など登録制のクラブで本サービスの仕組みを応用できると考える。
\section{今後に向けた課題}
運営上の課題は、現状では提供できるフットサルコートが少ないことと利用者への認知度が低いことである。フットサルコートを
今後は利用できる提携フットサルコートを増やした上で、利用者を徐々に増やしていく必要がある。
\section{サービス改善}
今後、本サービスを運営するにあたりに以下の2点が挙げられる。
\\\textbf{・集金システムの改善}
\\ チーム内での集金を試合当日に行うのではなく、本サービスで決済する前にチーム内で集金できるシステムがあれ代表者の決済の負担が軽減できると考え、株式会社手嶋屋のpne.clubを用いて、システムを採用してくれるチームを現在募集中である。
\\
\\\textbf{・コートごとに手数料を変える}
\\次にコートへの支払い金額についてである。Funフットサルクラブはフットサルスクールをメインの収入源にしていたため、自ら新規顧客を集約する力がなかったので、Livenupからの紹介手数料の意味合いを込めて、集金代金の50%を徴収していた。ただ東京都郊外のフットサル場には個人で参加し、即席チームで戦う個人フットサルで収入を得ているフットサルコートもある。個人の情報は多く保持しているが、前払い制ではないため当日にキャンセルされる可能もあるため本サービスの仕組みは有効である。このようにコートによって収入源が違うので、コートの経営形態に合わせた手数料を設定していく必要がると考える。


