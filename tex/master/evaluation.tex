\chapter{サービスの評価}
\section{調査概要}
本章では、2014年8月下旬から運営を開始して以降のサービスの運営内容を計測し、今後に向けての成果や課題を考察する。
\\ 定性調査として、本サービスの対象となるフットサルのコートのオーナー、プレイヤーへのインタビュー調査を実施することで、より客観的な調査結果を得ることができるように努めた。
またコートは本サービスで得た収入をどのような経営形態を変化させたのかを考察する。
\\ 本サービスを利用して頂いたコートのオーナーおよびフットサルチームのプレイヤーに本サービスの概要及び「購入ページ」、「予約ページ」、「コートの状況」といった各ページについてインタビュー調査を行った。
\section{調査結果:フットサルプレイヤー}
【サービス全体について】
\\プレイヤーから格安で試合ができ、試合相手を探す手間を省くことができたことが高い評価を得た。コートからは集金機能を本サービスが担うことで、集金する手間が省けるだけではなく、利益の損失するリスクを軽減できたことが評価された。
\\しかし、新サービス故の信頼性を指摘された。画面上部にある利用規約や運営者の情報を見やすい部分に設置することを指摘された
\\【購入ページについて】
\\会員制のページにしてほしい。一度、購入したら二度目以降はメールアドレスの記入等々を省いてほしい
\\【予約ページについて】
・カレンダー形式に表示するより、開催予定の日時だけを表示してほしい
【今後、本サービスに望むこと】
\\今後、本サービスに望まれていることは、下記3点である。
\\・会員制登録ページを作ってほしい
\\・同じ地域で運営する場合、最低3つは候補地があったほうがいいと指摘された
\\・淡々と試合をするだけでは、面白みに欠けるという意見出た。今後、モチベーションを上げるための景品などを用意したほうがいいという意見が出た




\section{調査結果:フットサルコートオーナー}
【サービス全体について】
\\コートからは集金機能を本サービスが担うことで、集金する利益の損失するリスクを軽減できたことと日程をエクセルシートで送付するだけで手間がないことが評価された。

【サービスによる収支について】
\\本サービスで一回分の試合をマッチングさせることで、コートはアルバイト1人分の日給に値する収入得ることが分かった。今後、これを週5日送客することで、アルバイト二人分の人件費に充てることができるとわかった。

【今後、本サービスに望むこと】
\\本サービスを用いて、フットサルコートの会員になるチームがなかった。今後、コートに会員登録してもらえるようにチームの人数を増やす仕組み、本サービスだけでコートに送客ができることを期待されている。言い換えるのであれば、サイト内のPV数を増やすことが望まれている。


\section{考察}
本節にてここまで記述してきたインタビュー調査についての結果について分析を実施する。
\\サービス利用者であるフットサルプレイヤーからは、価格面、サービスを通じて手軽にマッチングできる点から概ね高い評価を得ることができた。
今後、新規に利用者を獲得するには、紹介制を活用するだけではなく土地や年齢層にリーチできる広告媒体を選択し活用していくことが重要であると考える。
\\またサービス全体の規模としてはまだ東京都八王子での運営に留まっており、サービスの規模は小さくビジネスとして成立しているとは言い難い。
\\ 今後サービスを拡大していくには上記の利用者の母数を増やし、利用者のコメントを本サイト内で掲載し信頼を高めていくことに加えて、コートに定期的に送客できる仕組みを作ることが重要である。



